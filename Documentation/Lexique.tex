\documentclass[a4paper,10pt]{article}
\usepackage[utf8]{inputenc}
\usepackage[T1]{fontenc}
\usepackage[frenchb]{babel}
\frenchbsetup{StandardLists=true}
\usepackage{enumitem}
\usepackage{lmodern} % Pour changer le pack de police
\usepackage{makeidx}
%\usepackage{circuitikz}
%\usepackage{amsmath}
%\usepackage{amssymb}
%\usepackage{mathrsfs}
\usepackage{graphicx}
%\usepackage{url}
%\usepackage{array}

\title{Lexique \\	EZMeal}
\author{Groupe R\\ \\Jean-Benoît Berlier\\Laurent Desausoi \\ Martin Fockedey \\Maxime Mawait \\Van Stratum Arthur
}
\date{\today}
\makeindex

\begin{document}
\begin{titlepage}
\begin{figure}[t]
\includegraphics[scale=0.3]{epl-logo.jpg}
\end{figure}

\maketitle 
\end{titlepage}

\begin{description}
 \item [Auteur:] nom d'une personne ayant créé une recette. Ce nom n'est pas lié avec un user.
 \item [Avis:] ensemble facultatif constitué d'une note ainsi que d'un commentaire fait par un utilisateur sur une recette.
 \item [Commentaire:] texte unique fait par un utilisateur pour décrire son avis sur une recette.
 \item [Étapes:] fragments d'une recette, classés par ordre chronologique contenant diverses instructions pour mener à bien ladite recette
 \item [Ingrédient:] produit de base qui sera travaillé dans un recette, par exemble chocolat, sucre, huile, œuf,...
 \item [Recette:] ensemble d'instruction et d'ingrédient permettant de confectionner un plat.
 \item [Résidence:] adresse d'un utilisateur.
 \item [Sous-type:] attribut d'une recette tel que chaud, froid, italien, végétarien,... Peut ètre cumulé.
 \item [Type:] catégorie unique d'une recette tel qu'entrée, plat, dessert, boisson, etc.
 \item [User:] utilisateur de l'aplication.
 
\end{description}


\end{document}